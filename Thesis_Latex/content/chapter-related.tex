% !TEX root = ../thesis-example.tex
%
% Copypastas:
% "Text": \glqq Text\grqq{}
\chapter{Verwandte Arbeiten}
\label{sec:related}

In diesem Kapitel soll auf einige Arbeiten verwiesen werden, in denen Evaluationen von vorgestellten POS-Taggern stattfinden. Es wird besonders beachtet, \textit{wie} und nach welchen Metriken diese Ergebnisse berechnet werden.


\section{Stanford Tagger}
\label{sec:related:stanford}
In einem Artikel über den Stanford-Tagger, der zu POS-Tagging fähig ist, wird auch mit Daten von Sektionen aus dem annotierten \textit{Wall Street Journal} Korpus verglichen, um die Qualität der Taggingergebnisse zu prüfen. Hierzu werden unter Anderem die Metriken \textit{Per-Tag-Accuracy} und \textit{Per-Sentence-Accuracy} (siehe Kap. \ref{sec:concept:eval}) genutzt \\cite{Paper:StanfordTagger}.

\section{P. Paroubek: Evaluating Part-of-Speech Tagging and Parsing}

In dieser Arbeit (\cite{Paroubek})





%\section{Conclusion}
%\label{sec:related:conclusion}


