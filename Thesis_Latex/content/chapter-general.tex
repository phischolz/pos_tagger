% !TEX root = ../thesis-example.tex
%
% Copypastas:
% "Text": \glqq Text\grqq{}
\chapter{Allgemein}
\label{sec:general}

Um einheitliche Verarbeitung und Vergleichbarkeit zu ermöglichen, werden die Tags in \textit{Tagsets} definiert, wie zum Beispiel dem des Penn-Treebank-Projekts \linebreak \cite{Web:PennBank:2003}. \newline
Betrachtet man das Wort \glqq like \grqq{} aus dem einleitenden Beispiel \glqq I like the blue House.\grqq, dann fällt auf, dass es alternativ zum Verb \glqq mögen\grqq{} auch als Präposition \glqq wie\grqq{} interpretiert werden könnte, auch wenn der Satz dann keinen Sinn mehr ergibt. Diese Uneindeutigkeit (\textit{Ambiguität}) ist das zentrale zu lösende Problem für POS-Tagger  \cite{Smith:2011}; Im Gegensatz zum Menschen kann ein Algorithmus Ambiguitäten nicht intuitiv auflösen, sondern muss auf statistische, rechnerische und linguistische Methoden sowie Wissen aus diesen Teilbereichen zurückgreifen. \newline
%Goldstandards und Präzision

\section{Tagging-Ansätze}
\label{sec:general:types}
% "stumpf": nachschlagen und raten
% statistisch (ohne und mit kontext)
% regelbasiert
% lernend

\section{Probleme und Ziele}
\label{sec:generals:goals}



%\section{Related Work Section 2}
%\label{sec:related:sec2}
%
%
%\section{Related Work Section 3}
%\label{sec:related:sec3}
%
%
%
%\section{Conclusion}
%\label{sec:related:conclusion}


