% !TEX root = ../thesis-example.tex
%
\chapter{Allgemein}
\label{sec:general}

Um einheitliche Verarbeitung und Vergleichbarkeit zu ermöglichen, werden die Tags in \textit{Tagsets} definiert, wie zum Beispiel dem des Penn-Treebank-Projekts \linebreak \cite{Web:PennBank:2003}. \newline
Betrachtet man das Wort \textit{like} aus dem einleitenden Beispiel \textit{"I like the blue House".}, dann fällt auf, dass es alternativ zum Verb "mögen" auch als Präposition "wie" interpretiert werden könnte, auch wenn der Satz dann keinen Sinn mehr ergibt. Diese sog. \textit{Ambiguität} von Wörtern, also deren Mehrdeutigkeit, stellt das zentrale Problem des POS-Tagging-Prozesses dar \cite{Smith:2011}.

\section{Related Work Section 1}
\label{sec:related:sec1}

\Blindtext[2][2]

\section{Related Work Section 2}
\label{sec:related:sec2}

\Blindtext[3][2]

\section{Related Work Section 3}
\label{sec:related:sec3}

\Blindtext[4][2]

\section{Conclusion}
\label{sec:related:conclusion}

\Blindtext[2][1]
