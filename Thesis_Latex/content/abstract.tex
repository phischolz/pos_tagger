% !TEX root = ../thesis-example.tex
%
\pdfbookmark[0]{Abstract}{Abstract}
\chapter*{Abstrakt/Abstract}
\label{sec:abstract}
\vspace*{-10mm}

\subsubsection{Deutsch}
%Erklärende Einleitung NLP/POS
Im Themenbereich \textit{Natural Language Processing} (kurz NLP) versucht die Informatik, natürliche Sprachen für Algorithmen zugänglich und interpretierbar zu machen. Ein wichtiger Teil von NLP ist die Identifizierung der syntaktischen Bedeutung von Wörtern (\textit{Parts of Speech}, kurz POS) in gesprochener Sprache, und deren Zuweisung in Form von Tags (POS-Tagging). Für diese Aufgabe existiert eine Vielzahl unterschiedlich leistungsfähiger und robuster Algorithmen.
\newline
Im Rahmen dieser Arbeit wurde eine Sammlung solcher POS-Tagging-Algorithmen zusammen mit einem Evaluationssystem als Operatoren im Programm RapidMiner (zur Verfügung gestellt von RapidMiner GmbH) implementiert.

\subsubsection{English}

The field of Computer Science called \textit{Natural Language Processing} (NLP) attempts to make natural language accessible for machines and algorithms. One of the major parts of NLP is the identification of the syntactic roles of words and symbols (\textit{Parts of Speech, POS} in short) in spoken language, and tagging them as such POS (POS-Tagging). There is a multitude of differently capable and robust algorithms for this task.
\\
In this project, such POS-Tagging algorithms and a system to evaluate them is implemented in form of operators in an extension to the program RapidMiner (by RapidMiner GmbH).
