% !TEX root = ../thesis-example.tex
%
\chapter{Concepts}
\label{sec:eval}

Die folgenden POS-Tagger wurden implementiert:

\begin{table}[htb]
\begin{tabular}{l|l}
Tagger (Training Korpus) & Performance (Korpus)  \\
\cline{1-2}
NLP4J  & :TODO  	\\
LingPipe & :TODO 	\\
FastTag & -           
\end{tabular}
\caption{Liste der Implementierten Tagger. Performance laut eigenen Angaben.\\ \mbox{*} : Performance aus :NC }
\label{sec:eval:list}
\end{table}

Die Performance in Tab. \ref{sec:eval:list} bezieht sich auf eigene Angaben der Entwickler. Mit dem implementierten Evaluationsoperator sind wir allerdings in der Lage, die Tagger an einem eigenen annotierten Goldstandard-Korpus zu testen. Die folgenden Abschnitte diskutieren die Auswahl eines passenden Korpus und die Performance der Tagger auf diesem.

\section{Goldstandard-Korpus}
\label{sec:eval:corpus}

Ein annotierter Korpus muss dem Tagset der einzelnen Tagger entsprechen. Dies erweist sich als schwierig, da jeder Tagger u.U. Zeichen in unterschiedliche Token-Ketten spaltet und anders Kodiert. Es macht also Sinn, den Korpus und seine Tags auf den Tagger anzupassen, sofern das möglich ist, ohne den Informationsgehalt des Korpus zu ändern.



\section{Concepts Section 2}
\label{sec:concepts:sec2}



\section{Concepts Section 3}
\label{sec:concepts:sec3}



\section{Conclusion}
\label{sec:concepts:conclusion}


